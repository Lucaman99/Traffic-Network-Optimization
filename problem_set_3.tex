\documentclass{article}
\usepackage[utf8]{inputenc}

\title{Math Mentorship Problem Set \#3}
\author{Jack Ceroni}
\date{March 2019}

\begin{document}

\maketitle
\newpage
\section{Solutions}
1. To go about this question, let us first define what it means for us to be unable to label a point. To label a point $(a, \ b, \ c)$, we define another point $(a’, \ b’, \ c’) \ = \ f(a, \ b, \ c)$, where $f$ is some continuous function from the simplex to itself. In order to label a point $(a, \ b, \ c)$ as $1$, then $a’ \ < a$. Since we are looking for points that we cannot label, it is implied that $a’ \ \geq \ a$. Next, in order for a point to be labelled as $2$, we have $a’ \ \geq \ a$, but $b’ \ < \ b$. So this means that in order to not be able to label this point, $b’ \ \geq \ b$. Finally, to label a point as $3$, we have $a’ \ \geq \ a$, $b’ \ \geq \ b$, but $c’ \ < \ c$. Well, this then means that for our criteria, $c’ \ \geq \ c$. So we have found that in order to a point to not have labelled, all coordinates in the point $(a’, \ b’, \ c’)$ must be greater than or equal to their corresponding coordinates in $(a, \ b, \ c)$. Since $a \ + \ b \ + \ c \ = \ 1$, then $a’ \ + \ b’ \ + \ c’ \ \geq \ a \ + \ b \ + \ c \ = \ 1 \ \Rightarrow \ a’ \ + \ b’ \ + \ c’ \ \geq \ 1$. But it is also known that $a’ \ + \ b’ \ + \ c’$ must equal one, so $a’ \ + \ b’ \ + \ c’ \ = \ a \ + \ b \ + \ c$. Since each coordinate in the primed point is greater than or equal to its corresponding point in the unprimed point, this equality only occurs when $a \ = \ a’$, $b \ = \ b’$, and $c \ = \ c’$. So we have shown that the two points.
\newline\newline
2. Let us start with the point $(1, \ 0, \ 0)$. We must show that for any continuous function $f$, where $(a', \ b', \ c') \ = \ f(a, \ b, \ c)$ gives a labelling of $1$. To demonstrate this, we must show that $a'$ is always less than $a$. First off all, with obviously $a'$ cannot be greater than $1$ (by the definition of a simplex). Secondly, if we say that $a' \ = \ a$, and we know $a' \ + \ b' \ + \ c' \ = \ a \ + \ b \ + \ c \ = \ 1$, then $b' \ = \ b \ = \ 0$ and $c' \ = \ c \ = \ 0$. This means that for a mapping such as this $(a, \ b, \ c)$ is a fixed point. If we assume that $(a, \ b, \ c)$ is not a fixed point (as per the prior instructions), then the only possibility is that $a' \ < \ a$, and so by definition, the point $(1, \ 0, \ 0)$ has a labelling of $1$. We can apply the same logic to the other two cases. For $(0, \ 1, \ 0)$, we already know that $a' \ \geq \ a \ = \ 0$, by definition. Applying the exact same thought process as before, we come to conclusion that $b' \ < \ b$ and this point has a labelling of $2$. For the point $(0, \ 0, \ 1)$, we obviously know that $a' \ \geq \ a \ = \ 0$ and $b' \ \geq \ b \ = \ 0$, and by the previous logic, $c'$ must be less than $c$, and so this point has a labelling of $3$.
\newline\newline
3. First off, for some point $(a, \ b, \ 0)$, and some $f(a, \ b, \ 0) \ = \ (c, \ d, \ e)$, it is initially obvious that $(a, \ b, \ 0)$ cannot have a labelling of $3$, since we can't have some $e$ where $e \ < \ 0$. Now let $c \ \geq \ a$ and $d \ \geq \ b$. This means that $c \ + \ d \ + \ e \ \geq \ a \ + \ b \ = \ 1$, and because of the previous fact, this can only hold true upon equality ($c \ = \ a$ and $d \ = \ b$), meaning $e \ = \ 0$, showing that this can be a fixed point. In addition, there will be mappings that result in $1$ and $2$, since we are not imposing constraints that defy the definition of the simplex itself (as they would be if a labelling of $3$ was possible for this point).
\newline\newline
4. To demonstrate this fact for $(0, \ 1, \ 0)$ and $(0, \ 0, \ 1)$, we apply the exact same logic as before, except in this case, we can't have some $c \ < \ a$, since $a$ is always equal to $0$, meaning that a point lying on this edge must be fixed, or labelled $2$ or $3$. Similarly, a point on the edge connecting the points $(1, \ 0, \ 0)$ and $(0, \ 0, \ 1)$ must be fixed, or labelled $1$ or $3$.
\newline\newline
5. First, we can assume that there are no fixed points on the edges, otherwise the proof would already be done. This simplex is a Sperner labelling because it adheres exactly to the rules for creating a Sperner labelling through what we have already proved. Each of the vertices, $(1, \ 0, \ 0)$, $(0, \ 1, \ 0)$, $(0, \ 0, \ 1)$ are labelled $1$, $2$, and $3$. We have also shown that points on the edges (since we have said they aren't fixed point) have a labelling corresponding to one of the vertices forming the edge. This is the exact definition of a Sperner labelling.
\newline\newline
6. Let us take $T_0$ to be the largest triangle, where the vertices are labelled $1$, $2$, and $3$. First, let us come to the conclusion that on the edged of the triangle $T_0$, excluding the vertices, there must be \textbf{at least} two different labellings of points. Let us define points $x$ and $y$ lying on edges of the triangle that give distinct labellings, if there are any two distinct labellings. We can then draw a line between $x$ and $y$, and along with the vertex that does not have a labelling corresponding to $x$ or $y$ and we have a triangulation with labelling $(1, \ 2, \ 3)$. If each side of the triangle has only one distinct labelling (so one side has all points labelled $1$, one has all points labelled $2$, and one has all points labelled $3$), we can draw lines between three points lying on each of these edges and we have the desired triangle. If there is a "mixture" of labellings on any of the edges, we can chose one of the previous methods to create the triangulation. Now we have a $T_1$ that has a labelling of $(1, \ 2, \ 3)$. We can repeat this process and create a $T_2$ such that we again get a triangle with labelling $(1, \ 2, \ 3)$ through the exact same process. We can continue to iterate for any $T_n$, therefore proving this statement.
\newline\newline
\end{document}
